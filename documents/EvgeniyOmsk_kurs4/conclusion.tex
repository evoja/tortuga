\chapter*{Заключение}						% Заголовок
\addcontentsline{toc}{chapter}{Заключение}	% Добавляем его в оглавление

Web–приложение «Тортуга» создано для обучения школьников основам программирования, и для достижения максимального результата этот процесс должен быть удобен как ученику, при выполнении заданий, так и учителю, при формировании урока. Длинные ссылки вида (2), которыми затруднительно делиться, использовать в презентациях, размещать в соц–сетях, отсутствие инструмента для редактирования готовых уроков, отсутствие функции очистки экрана являлись слабым местом приложения, требующие исправления.\par

Возможность  автоматизации сборки веб-приложения обладает самым крупным потенциальным резервом для повышения эффективности разработки, снижения требуемых материальных и трудовых ресурсов, сокращения монотонной работы, повышения производительности разработчиков и качества выпускаемого продукта. Тем самым ускорив его разработку.\par
\vspace{16mm}
Достигнутые результаты:
\begin{itemize}
  \item Сокращение ссылок
  \item Очистка экрана clearCanvas
  \item Изменение толщины рисования
  \item Редактирование уроков
\end{itemize}


\clearpage