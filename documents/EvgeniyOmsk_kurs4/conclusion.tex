\chapter*{Заключение}						% Заголовок
\addcontentsline{toc}{chapter}{Заключение}	% Добавляем его в оглавление

Web-приложение «Тортуга» создано для обучения школьников основам программирования, и для достижения максимального результата этот процесс должен быть удобен как ученику, при выполнении заданий, так и учителю, при формировании урока. Длинные ссылки вида (2), которыми затруднительно делиться, использовать в презентациях, размещать в соц-сетях, отсутствие инструмента для редактирования готовых уроков, отсутствие функции очистки экрана являлись слабым местом приложения, требующие исправления.\par

Возможность  автоматизации сборки веб-приложения обладает самым крупным потенциальным резервом для повышения эффективности разработки, снижения требуемых материальных и трудовых ресурсов, сокращения монотонной работы, повышения производительности разработчиков и качества выпускаемого продукта. Тем самым ускорив его разработку.\par
\vspace{16mm}
Достигнутые результаты:
\begin{itemize}
  \item Реализовано сокращение ссылок
  \item Добавлена функция очистки экрана
  \item Изменение толщины рисования
  \item Написан модуль по редактированию уроков
  \item Добавлено изменение вида концов толстых линий
  \item Реализовано перестроение урока без перезагрузки
  \item Добавлена возможность изменения координат и угла поворота
  \item Написан модуль обработки событий мыши
  \item Поддержка перемещенных файлов
  \item Реализована автоматизация сборки
  \item Произведен рефакторинг архитектуры
  
\end{itemize}


\clearpage